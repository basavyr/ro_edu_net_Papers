\documentclass[journal]{IEEEtran}

\usepackage{lipsum}

% correct bad hyphenation here
\hyphenation{op-tical net-works semi-conduc-tor}

\begin{document}

\title{\texttt{pyxrootd} \& C++ Declarative API \\within \texttt{XROOTD}}
\author{Robert Poenaru, Michał Simon}

\maketitle

\begin{abstract}
A brief description of the \texttt{xrootd} architecture and its purpose within the WLCG group, together with an overview of the server- and client- sides of the xrootd framework are provided within the present work. The client side of xrootd has a relatively new implementation called Declarative API. Its main purpose is to provide the user an asyncrhonous interface that is more inline with the modern C++ paradigm. A focus on the development workflow for this API is given. Moreover, the \texttt{pyxrootd} package, which provides a python interface with the xrootd client, is also discussed and tested in a usual file-operation usecase.
\end{abstract}

% Note that keywords are not normally used for peerreview papers.
\begin{IEEEkeywords}
xrootd, pyxrootd, asyncrhonous programming, decalarative API, pipeline, server, client.
\end{IEEEkeywords}

\IEEEpeerreviewmaketitle

\section{Introduction}
\lipsum[1]

\subsection{Subsection Heading Here}
Subsection text here.
\lipsum[1-2]
% needed in second column of first page if using \IEEEpubid
% \IEEEpubidadjcol

\section{Conclusion}
The conclusion goes here.

\appendices
\section{}
Appendix one text goes here.

\section{}
Appendix two text goes here.


% use section* for acknowledgment
\section*{Acknowledgment}
The authors would like to thank...

% Can use something like this to put references on a page
% by themselves when using endfloat and the captionsoff option.
\ifCLASSOPTIONcaptionsoff
  \newpage
\fi

\begin{thebibliography}{1}

\bibitem{IEEEhowto:kopka}
H.~Kopka and P.~W. Daly, \emph{A Guide to \LaTeX}, 3rd~ed.\hskip 1em plus
  0.5em minus 0.4em\relax Harlow, England: Addison-Wesley, 1999.
\end{thebibliography}

\end{document}


